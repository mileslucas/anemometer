\section{Introduction} \label{sec:intro}

The last twenty years of astronomy have seen a revolution in planetary science, with thousands of exoplanets discovered around nearby stars. While our understanding of exoplanet demographics has lept forward in recent years, fundamental questions remain. What are the dominant planet formation pathways? How do planetary atmospheres form and evolve? Is there life on other planets?

These questions can only be answered through the spectroscopic characterization of exoplanets over a range of masses and orbital separations from their host stars. While each method of exoplanet detection plays an important role in surveying planets, only the methods of transits and direct imaging allow for the possibility of spectroscopic characterization. Directly imaging exoplanets is versatile- the planet's orbit does not have to transit its host star and the orbital period is less crucial for detection compared to radial velocity, transit, and astrometric methods \citep{2013pss3.book..489W}. The challenge in direct imaging, however, is appropriately separating and nulling the starlight to detect the dim exoplanet signal \citep{2010exop.book..111T}.

The field of direct imaging (or high-contrast imaging) comprises modern instrumentation, observational techniques, and post-processing algorithms to overcome the obstacles in imaging a planet outside our solar system. Adaptive optics (AO) play a critical role in direct imaging by stabilizing the stellar point-spread function (PSF), enabling modern coronagraphs to attenuate the stellar signal \citep{2005ApJ...629..592G}. AO systems consist of a wavefront sensor, deformable mirror, and a real-time controller. The controller takes the images from the wavefront sensor and calculates the deformable mirror commands. This loop needs to run at kHz speeds to enable high-contrast imaging \citep{2018ARA&A..56..315G}.

Typically, the AO system is ``upstream" of the science instrument, which means that any wavefront errors ``downstream" of the wavefront sensor cannot be corrected (e.g., a polishing defect). These wavefront errors are referred to as non-common path aberrations. The aberrations accumulate in the focal plane as quasi-static speckles, which are aberrations of the true PSF \citep{2018ARA&A..56..315G}. The speckles have structure, but it is variable due to slow-changing effects such as thermal gradients. This variability leads to their quasi-static nature.

AO systems are also constrained by engineering and runtime factors such as the density of deformable mirror segments, and the speed of the real-time control. These factors limit how well an AO system can correct a wavefront. For ground-based telescopes these limits are significant due to high-speed atmospheric turbulence variation, which evolves faster than the AO system can correct, leading to residual wavefront errors \citep{Soummer_2007}. These errors manifest in fast speckles with coherence times $\sim$0.01s, which is much less than a typical science integration ($\sim$10-100s). This causes the speckles to average out over an integration and form a halo with exponentially decreasing brightness from its center. The quasi-static speckles and halo are hard to distinguish from astrophysical signals, limiting our sensitivity to exoplanets (see \autoref{fig:wdh}).

\begin{figure}
    \centering
    % \epsscale{0.7}
    \plotone{wdh}
    \caption{Coronagraphic focal-plane images showing the wind-driven halo effect. The left figure is a simulated model of the SPHERE-IRDIS instrument with a wind-driven halo. The right figure is an H2-band exposure from SPHERE-IRDIS. The circled regions in the right figure are known optical artifacts. The existence of the wind-driven halo greatly compromises the sensitivity for detecting exoplanets in high-contrast imaging. Adapted from \cite{2020AA...638A..98C}.}
    \label{fig:wdh}
\end{figure}

\subsection{Predictive wavefront control} \label{sec:pwfc}

Predictive wavefront control improves on current AO control laws (the math that determines the DM commands given the wavefront sensor measurement) by \textit{predicting} the future wavefront errors. Predicting the future wavefront would allow extreme AO systems to avoid the time-lag error which creates the wind-driven halo in the focal plane images \citep{2018ARA&A..56..315G}. One of the algorithms for predictive control is \textit{Empirical Orthogonal Functions} (EOF; \citealp{guyon_adaptive_2017}), which uses a linear combination of previous wavefront measurements (which have been projected onto an orthogonal subspace) to predict the future wavefront. This method is theoretically well-suited to counteract the effects of wind in the atmosphere, since we believe wind is comprised of linear flows (the \textit{frozen-flow hypothesis}). \citet{guyon_adaptive_2017} reports an expected $\sim$100 times improvement in root-mean-square (RMS) wavefront error using EOF, however on-sky performance is only a factor of a few improved.

\begin{figure}
    \centering
    \epsscale{0.8}
    \plotone{eof-performance}
    \caption{Demonstration of predictive control for high-contrast imaging. The following panels show simulated focal-plane PSF: (a) raw PSF contrast (scale divided by the maximum value), (b) contrast with coronagraphic corrections; here the fast-evolving speckle halo is clear, and (c) contrast with predictive control and coronagraphic corrections. Imaging a Jupiter analogue requires contrast of $\sim10^{-8}$ and for an Earth analogue in reflected light is $\sim10^{-10}$, for reference.}
    \label{fig:eof-perf}
\end{figure}

The shortcoming of predictive control on-sky is an active topic of adaptive optics research \citep{2018ARA&A..56..315G} and is paramount to the success of large-aperture ($>$\SI{10}{\meter}) telescopes for exoplanet imaging. Why do the on-sky tests perform so much worse than the testbed results? Is the frozen-flow hypothesis valid in the regime of extreme AO? These questions are key to answer in the coming years in order to push ground-based astronomy further in exoplanet research.

\subsection{Studying wind speed} \label{sec:windspeed}

Wind speed is important to study in order to improve current predictive control methods. Alluded to in \autoref{sec:pwfc}, the frozen-flow hypothesis of linear wind flows is not guaranteed to be true, and discrepancies may be part of why predictive control does not perform as well on-sky as in theory. In addition, part of the EOF control algorithm requires training a low-rank orthogonal basis to wavefront data. If the flows (in the frozen-flow hypothesis) evolve, by changing direction or speed, this leads to a dramatic decrease in the performance of the EOF law. This means in current applications predictors are retrained every 30 minutes, but a more thorough study of the dynamics of wind speed (how much it changes, and how rapidly) would greatly optimize the application and performance of predictive control. In fact, developing an \textit{online} algorithm that can be run alongside the EOF control law to estimate when to retrain would greatly improve the stability of the wavefront control.

This report details our how we can use AO telemetry to measure wind speed (\autoref{sec:methods}), including the details of our algorithm (\autoref{sec:algo}), the verification of the algorithm in simulated and testbed experiments (\autoref{sec:simulated}, \autoref{sec:turbulence}), and the current status of on-sky application of the algorithm (\autoref{sec:onsky}).