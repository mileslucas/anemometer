\section{Conclusions} \label{sec:conclusions}

Improving predictive control methods for extreme AO is a necessary and prescient problem as ground-based telescope diameters grow larger and the push to discover Earth-like exoplanets increases. In this report, we outlined the shortcomings of predictive control and specifically how wind negatively affects the performance of high-contrast imaging. We designed a study to improve our knowledge of the dynamics of how wind evolves in the pupil plane. We created an algorithm for estimating wind speed using pseudo-open-loop data and validated it on simulated data and on the SCExAO testbed. This algorithm is not yet suited for on-sky data, which suffer from many more complications and are generally noisier than injecting simulated data on the testbed. Our results only recover the apparent motion in on-sky data $\sim$20\% in the data we tested with. Before we can try to analyze wind dynamics we need to find appropriate pre-processing steps to maximize the signal of the motion in the cross-correlation. We also need to validate that the pre-processing is applicable to different data with different apparent motion, otherwise we may bias the speed measurements implicitly.